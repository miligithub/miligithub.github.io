%-------------------------------------------------------------------------------
%	SECTION TITLE
%-------------------------------------------------------------------------------
\vspace{-.15in}
\cvsection{Projects}
\vspace{-.05in}

%-------------------------------------------------------------------------------
%	CONTENT
%-------------------------------------------------------------------------------
\begin{cventries}
	
	%---------------------------------------------------------
	\cvprojectentry
	{2019-2020 SpyPhone}
	{Eavesdropping on Smartphone Speakers with Motion Sensors}
	{
		\begin{cvitems} % Description(s) of tasks/responsibilities
			\item Identified a security issue on smartphones that motion sensors (access granted to any app) can eavesdropping on speakers.
			\item Developed an Android app to collect motion sensor (accelerometer and gyroscope) data while playing sounds through speakers.
			\item Reconstructed the high frequency (16,000 Hz) sound information from  low frequency (400 Hz) motion data by building K-SVD dictionaries and recognized the speech using Bi-LSTM networks.
			\item  \underline{Utilized}: Recurrent Neural Networks, Compressed Sensing, Android Programming.
		\end{cvitems}
	}
	
	
	%---------------------------------------------------------
	\cvprojectentry
	{2018-2019 MotionVoice}
	{A Spoof-proof Voice Authentication System for Smartphones}
	{
		\begin{cvitems} % Description(s) of tasks/responsibilities
			\item Proposed a new voice authentication system that is immune to replay attacks by leveraging the self demodulation and acoustic attenuation effect when sound signals transmitted through human body.
			\item Designed an Android app to collect the body-borne vibration and sound data simultaneously.
			\item Applied signal processing techniques such as syllable separation to sound data and using sequence-to-sequence LSTM network on vibration data to identify users.
			\item  \underline{Utilized}: Machine learning, Digital Signal Processing, Android Programming.
		\end{cvitems}
	}
	
	
	%---------------------------------------------------------
	\cvprojectentry
	{2018 DriverDetect}
	{Using Atmospheric Pressure Sensors to Determine Whether the User is the Driver or a Passenger}
	{
		\begin{cvitems} % Description(s) of tasks/responsibilities
			\item Proposed a new driver detection system based on the fact that each seat in the vehicle is expected to experience differences in atmospheric pressure for each vehicle acceleration dynamic. 
			\item Designed a circuit with Arduino to test the system. 
			\item \underline{Utilized}: Arduino programming, Digital Signal Processing.
		\end{cvitems}
	}
	
	
	
%	%---------------------------------------------------------
%	\cvprojectentry
%	{UltraUnlock}
%	{A Sequence-based Place Recognition Method}
%	{
%		\begin{cvitems} % Description(s) of tasks/responsibilities
%			\item Computer Vision, Android Programming
%		\end{cvitems}
%	}
	
	
	%---------------------------------------------------------
	\cvprojectentry
	{2016-2017 QUAC}
	{Quality-Aware Contract-Based Incentive Mechanisms for Crowdsensing}
	{
		\begin{cvitems} % Description(s) of tasks/responsibilities
			\item Design two quality-aware contract-based incentive mechanisms for crowdsensing, named QUAC-F and QUAC-I, under full information model and incomplete information model, respectively, which differ in the level of users’ information known to the system.
			\item Mathematically proved that both QUAC-F and QUAC-I are guaranteed to maximize the platform utility while satisfying individual rationality and incentive compatibility.
			\item  \underline{Utilized}: Algorithm Design, Contract Theory.
		\end{cvitems}
	}
	
%	%---------------------------------------------------------
%	\cvprojectentry
%	{RecPlace}
%	{A Sequence-based Place Recognition Method}
%	{
%		\begin{cvitems} % Description(s) of tasks/responsibilities
%			\item Computer Vision, Android Programming.
%		\end{cvitems}
%	}
	
	%---------------------------------------------------------
	\cvprojectentry
	{2016 TurtleBot}
	{Using RTAB-Map and a TurtleBot to Create a Floor Map}
	{
		\begin{cvitems} % Description(s) of tasks/responsibilities
			\item Used a TurtleBot to perform graph-based simultaneous localization and mapping (SLAM) by using RTAB-Map.
			\item Programmed speech control and voice feedback on the TurtleBot.
			\item  \underline{Utilized}: ROS Robot Programming, Computer Vision.
		\end{cvitems}
	}
	
	
	
	
	%---------------------------------------------------------
	\cvprojectentry
	{2015-2017 SpecWatch} % Organization
	{Solving Adversasial Spectrum Usage Monitoring Problem with Unknown Statistics in CRNs.} % Job title
	%    {Mines, Golden, CO} % Location
	%    {Jan. 2014 - May. 2015} % Date(s)
	{
		\begin{cvitems} % Description(s) of tasks/responsibilities
			\item Modeled the monitoring problem as an adversarial multi-armed bandit problem with switching cost.
			\item
			Designed an asymptotically optimal online algorithm, termed {\textsc{SpecWatch}}, and prove its normalized expected weak regret is $O(1/\sqrt[3]{T})$, which converges to 0 as time horizon $T$ approaches to $\infty$. 
%			\item
%			Improve the performance of {{SpecWatch}} and design another algorithm, {{SpecWatch+}}, which reduces the bias in selecting channels and explores all channels more efficiently. 
			%	This algorithm guarantees the actual value of weak regret to be  $O(T^{2/3})$, which is asymptotically optimal as well, with probability $1 - \delta$, for any $\delta \in (0,1)$.	
%			\item
%			Program and run extensive simulations whose results support the theoretical findings.
			\item  \underline{Utilized}: Algorithm Design, Game Theory.
		\end{cvitems}
	}
	
	%---------------------------------------------------------
	\cvprojectentry
	{2014-2015 IntelliSample} % Organization
	{Self-tuning Program to Output the Shortest Paths Efficiently on Very Large Graphs} % Job title
	%    {Mines, Golden, CO} % Location
	%    {Aug. 2014 – Dec. 2014} % Date(s)
	{
		\begin{cvitems} % Description(s) of tasks/responsibilities
			\item Implemented three shortest path algorithms (Bellman Ford's, Dijkstra's, and Gabow's) and two sampling methods based on Forest Fire Algorithm.
			\item Provided a framework which predicts the best shortest path algorithm by pre-running on the sample graph.
			\item  \underline{Utilized}: Algorithm Selection, Graph Sampling.
		\end{cvitems}
	}
	
	%---------------------------------------------------------
	\cvprojectentry
	{2014 PaperSelect} % Organization
	{Program to Automatically Select Academic Papers for Researchers} % Job title    
	%    {Mines, Golden, CO}% Location
	%    {Jan. 2014 – May. 2014 } % Date(s)
	{
		\begin{cvitems} % Description(s) of tasks/responsibilities
%			\item Be the first to formulate the paper selection problem.
			\item Provided two greedy approaches to solve the problem, one is heuristic and the other is $(1$$-$$1/e)$-approximate. 
			\item  \underline{Utilized}: Algorithm Design.	
		\end{cvitems}
	}
	
	
%---------------------------------------------------------
\end{cventries}
